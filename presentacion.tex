%! Author = Xandra Campo Blanco
%! Date = 25/2/25

% Preamble
\documentclass{beamer}
\usetheme[secheader]{Madrid}

% Packages
%\usepackage{amsmath}
\usepackage[dvipsnames]{xcolor}

% Commands
\newcommand{\highlight}[1]{{\color{Blue} #1}}

\title[Prevención frente al acoso sexual]{Prevención y actuación temprana frente al acoso sexual y por razón de sexo, incluido en el ámbito digital}
\subtitle{Plan de formación 2025 \newline Subdirección General de RRHH e Inspección de Servicios \newline Ministerio de Ciencia, Innovación y Universidades}
\author[X. Campo]{Xandra Campo}
\institute[CIEMAT]{CIEMAT}
\date{25 de febrero de 2025}

\AtBeginSection{%
    \begin{frame}
        \tableofcontents[sections=\value{section}]
    \end{frame}
}

% Document
\begin{document}
    \maketitle
    \begin{frame}
        \frametitle{Prevención y actuación temprana frente al acoso sexual y por razón de sexo, incluido en el ámbito digital}
		\tableofcontents[hideallsubsections]
	\end{frame}

    \section{Sobre el curso}

    \subsection{Información general}

    \begin{frame}
		\frametitle{Sobre el curso}
        \framesubtitle{Información general}
        Organización:
        \begin{description}[Other description]
            \item[Organizador] Subdirección General de RRHH e Inspección de Servicios del MICIU
            \item[Destinatarios] Personal del MICIU y organismos dependientes
            \item[Plan de formación] 2025
            \item[Área formativa] Igualdad de Género
            \item[Modalidad] Videoconferencia
            \item[Duración] 9 horas
            \item[Días y horario] 20 - 24 de febrero, de 10:00 a 13:30
        \end{description}
        Profesorado:
        \begin{description}[Other description]
            \item[Mar Liñán] Responsable del Servicio de PRL del MICIU
            \item[Ignacio Cudeiro] Inspector General de Servicios del MICIU
            \item[Paz Lloria] Catedrática de Derecho penal, Universitat de València
        \end{description}
    \end{frame}

    \subsection{Antecedentes y motivación}

    \begin{frame}
		\frametitle{Sobre el curso}
        \framesubtitle{Antecedentes y motivación}
            Leyes relevantes:
            \begin{itemize}
            \item \highlight{Ley Orgánica 3/2007} para la igualdad efectiva de hombres y mujeres.
            \item \highlight{Ley Orgánica 10/2022} de garantía integral de la libertad sexual.
            \item \highlight{Ley 17/2022} modifica la Ley 14/2011 de Ciencia, Tecnología e Innovación.
            \end{itemize}
            Requerimientos:
            \begin{itemize}
            \item \highlight{Informar, sensibilizar y formar} al personal sobre acoso y violencia de género, incluyendo el ámbito digital y la I+D+I.
            \item Los agentes públicos de SECTI deben implementar \highlight{Planes de igualdad de género} y protocolos contra el acoso con seguimiento anual.
            \end{itemize}
    \end{frame}

    \begin{frame}
		\frametitle{Sobre el curso}
        \framesubtitle{Antecedentes y motivación}
            Compromiso del ministerio:
            \begin{itemize}
            \item \highlight{Erradicación del acoso sexual y por razón de sexo}: Apoyo a universidades y centros de investigación para crear entornos laborales diversos, inclusivos, seguros e igualitarios.
            \item \highlight{Formación adecuada}: Garantizar que todo el personal esté capacitado para prevenir, detectar y abordar situaciones de acoso sexual o por razón de sexo, también en el ámbito digital.
            \end{itemize}
            Contexto actual:
            \begin{itemize}
            \item \highlight{Persistencia del acoso}: El acoso sexual y por razón de sexo sigue siendo un problema en entornos laborales jerarquizados.
            \item \highlight{Impacto de las TIC}: La revolución digital ha influido en las desigualdades de género y la violencia contra las mujeres, incluyendo el acoso sexual y por razón de sexo.
            \end{itemize}
    \end{frame}

    \subsection{Objetivos}

    \begin{frame}
		\frametitle{Sobre el curso}
        \framesubtitle{Objetivos}
        Proporcionar una \highlight{formación adecuada} desde el punto de vista preventivo frente a:
        \begin{itemize}
            \item Situaciones de \highlight{acoso sexual y acoso por razón de sexo} como riesgo psicosocial en el ámbito laboral:
            \begin{itemize}
                \item En qué consiste y medidas administrativas y disciplinarias para prevenir
                \item Detección temprana y actuación dentro del protocolo de aplicación
            \end{itemize}
            \item Situaciones de desigualdad de poder en el \highlight{ámbito digital}:
            \begin{itemize}
                \item Normativa existente al respecto
                \item Recomendaciones prácticas para ayudar a evitar, detectar y erradicar
            \end{itemize}
        \end{itemize}
    \end{frame}

    \section{AS y ARS como riesgo psicosocial en el ámbito laboral}

    \subsection{Normativa aplicable en materia de AS y ARS}

    \begin{frame}
        \frametitle{Normativa aplicable en materia de AS y ARS}
        \framesubtitle{Normativa internacional}
        \begin{itemize}
            \item \textbf{Convenio 190} \textit{(OIT, 2019) Eliminación de la violencia y el acoso en el mundo del trabajo}:
            \begin{itemize}
                \item Primera norma para prevenir y eliminar la violencia y el acoso en el trabajo, incluyendo el acoso por razón de género
                \item Ratificado por España en 2022 y en vigor desde 2023
                \item Protege a todas las personas en el mundo del trabajo, independientemente de su relación contractual y abarcando diversos contextos laborales
            \end{itemize}
        \end{itemize}
    \end{frame}

    \begin{frame}
        \frametitle{Normativa aplicable en materia de AS y ARS}
        \framesubtitle{Normativa europea}
        \begin{itemize}
            \item \textbf{Directiva 2006/54/CE} \textit{Igualdad de trato entre hombres y mujeres en materia de empleo y ocupación}: Prohíbe el AS y ARS y exige sanciones efectivas y reparación a las víctimas
            \item \textbf{Directiva 2019/1158} \textit{Conciliación de la vida familiar y profesional}: Promueve un entorno laboral inclusivo y protege contra la discriminación por razón de sexo
            \item \textbf{Convenio de Estambul} \textit{(2011) Prevención y lucha contra la violencia contra las mujeres y la violencia doméstica}: Reconoce el AS como violencia de género y exige medidas preventivas y sancionadoras.
            \item \textbf{Carta de los Derechos Fundamentales de la UE}: Prohíbe la discriminación por razón de sexo y exige igualdad entre hombres y mujeres.
        \end{itemize}
    \end{frame}

    \begin{frame}
        \frametitle{Normativa aplicable en materia de AS y ARS}
        \framesubtitle{Normativa española}
        \begin{itemize}
            \item \textbf{LO 3/2007} \textit{Igualdad efectiva de mujeres y hombres}: Define y sanciona el AS y ARS como discriminación
            \item \textbf{RD 901/2020} \textit{Planes de igualdad y registro}: Protocolos específicos contra AS y ARS en planes de igualdad
            \item \textbf{RD 902/2020} \textit{Igualdad retributiva de mujeres y hombres}: Relaciona discriminación salarial y desigualdad de género
            \item \textbf{LO 10/2022} \textit{Garantía integral de la libertad sexual}: Protección contra AS en cualquier ámbito, exige sensibilización y formación
            \item \textbf{Ley 31/1995} \textit{Prevención de Riesgos Laborales}: Define el acoso como riesgo psicosocial, obliga a la prevención
            \item \textbf{LO 10/1995} \textit{Código Penal}: Regula el AS y ARS como delito
            \item \textbf{RDL 2/2015} \textit{Estatuto de los Trabajadores}: Prohibición, derechos y medidas contra el AS y ARS
            \item \textbf{RDL 5/2015} \textit{Estatuto Básico del Empleado Público}: Prohíbe y sanciona el AS y ARS
        \end{itemize}
    \end{frame}

    \subsection{Visión del AS y ARS desde la perspectiva de la PRL}

    \begin{frame}
		\frametitle{AS y ARS como riesgo psicosocial en el ámbito laboral}
        \framesubtitle{Visión del AS y ARS desde la perspectiva de la PRL. Riesgos psicosociales}
	\end{frame}

    \subsection{Definiciones}
    \begin{frame}
		\frametitle{AS y ARS como riesgo psicosocial en el ámbito laboral}
        \framesubtitle{Definiciones}
	\end{frame}

    \subsection{Percepción social del acoso sexual y acoso por razón de sexo}
    \begin{frame}
		\frametitle{AS y ARS como riesgo psicosocial en el ámbito laboral}
        \framesubtitle{Percepción social del acoso sexual y acoso por razón de sexo}
	\end{frame}

    \subsection{Perfil de víctima y acosador/a}
    \begin{frame}
		\frametitle{AS y ARS como riesgo psicosocial en el ámbito laboral}
        \framesubtitle{Perfil de víctima y acosador/a}
	\end{frame}

    \subsection{Consecuencias del acoso sexual y acoso por razón de sexo}
    \begin{frame}
		\frametitle{AS y ARS como riesgo psicosocial en el ámbito laboral}
        \framesubtitle{Consecuencias del acoso sexual y acoso por razón de sexo}
	\end{frame}

    \subsection{Asistencia/apoyo a las víctimas de acoso}
    \begin{frame}
		\frametitle{AS y ARS como riesgo psicosocial en el ámbito laboral}
        \framesubtitle{Asistencia/apoyo a las víctimas de acoso}
	\end{frame}

    \subsection{Detección temprana del acoso sexual y acoso por razón de sexo}
    \begin{frame}
		\frametitle{AS y ARS como riesgo psicosocial en el ámbito laboral}
        \framesubtitle{Detección temprana del acoso sexual y acoso por razón de sexo}
	\end{frame}

    \subsection{Prevención del acoso sexual y acoso por razón de sexo}
    \begin{frame}
		\frametitle{AS y ARS como riesgo psicosocial en el ámbito laboral}
        \framesubtitle{Prevención del acoso sexual y acoso por razón de sexo}
	\end{frame}

	\subsection{Proyectos}
    \begin{frame}
		\frametitle{AS y ARS como riesgo psicosocial en el ámbito laboral}
        \framesubtitle{Proyectos}
	\end{frame}

    \section{Protocolo frente al acoso sexual y acoso por razón de sexo}

    \subsection{Normativa}
    \begin{frame}
		\frametitle{Protocolo frente al acoso sexual y acoso por razón de sexo}
        \framesubtitle{Normativa}
	\end{frame}

    \subsection{Definiciones}
    \begin{frame}
		\frametitle{Protocolo frente al acoso sexual y acoso por razón de sexo}
        \framesubtitle{Definiciones}
	\end{frame}

    \subsection{Ámbito penal}
    \begin{frame}
		\frametitle{Protocolo frente al acoso sexual y acoso por razón de sexo}
        \framesubtitle{Ámbito penal}
	\end{frame}

    \subsection{Ámbito administrativo}
    \begin{frame}
		\frametitle{Protocolo frente al acoso sexual y acoso por razón de sexo}
        \framesubtitle{Ámbito administrativo}
	\end{frame}

    \subsection{Medidas administrativas en el MICIU}
    \begin{frame}
		\frametitle{Protocolo frente al acoso sexual y acoso por razón de sexo}
        \framesubtitle{Medidas administrativas en el MICIU}
	\end{frame}

    \subsection{Medidas disciplinarias}
    \begin{frame}
		\frametitle{Protocolo frente al acoso sexual y acoso por razón de sexo}
        \framesubtitle{Medidas disciplinarias}
	\end{frame}

    \section{Acoso y violencia de género en el ámbito digital}

    \subsection{Problemas a abordar en violencia digital sexual}
    \begin{frame}
		\frametitle{Acoso y violencia de género en el ámbito digital}
        \framesubtitle{Problemas a abordar en violencia digital sexual. Punto de vista social y criminológico}
	\end{frame}

    \subsection{Dificultades en la investigación de delitos tecnológicos de género}
    \begin{frame}
		\frametitle{Acoso y violencia de género en el ámbito digital}
        \framesubtitle{Dificultades en la investigación de delitos tecnológicos de género}
	\end{frame}

    \subsection{Exigencias de la nueva directiva de violencia}
    \begin{frame}
		\frametitle{Acoso y violencia de género en el ámbito digital}
        \framesubtitle{Exigencias de la nueva directiva de violencia}
	\end{frame}

    \section{}
    \subsection{}

	\begin{frame}
        \begin{block}{}
			\centering
			¡Gracias por vuestra atención!
		\end{block}
	\end{frame}

\end{document}