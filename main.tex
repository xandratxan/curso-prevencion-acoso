%! Author = Xandra Campo Blanco
%! Date = 6/4/25

% Preamble
\documentclass{beamer}
\usetheme[secheader]{Madrid}
\usecolortheme{dolphin}
\setbeamertemplate{navigation symbols}{}

\title[Prevención frente al acoso sexual]{Prevención y actuación temprana frente al acoso sexual y por razón de sexo, incluido en el ámbito digital}
\subtitle{Plan de formación 2025 \newline Subdirección General de RRHH e Inspección de Servicios \newline Ministerio de Ciencia, Innovación y Universidades}
\author[X. Campo]{Xandra Campo}
\institute[CIEMAT]{CIEMAT}
\date{Abril de 2025}

% Packages
\usepackage{amsmath}

\AtBeginSection{%
    \begin{frame}
        \tableofcontents[sections=\value{section}]
    \end{frame}
}

% Document
\begin{document}
    \maketitle
    \begin{frame}
        \frametitle{Prevención y actuación temprana frente al acoso sexual y por razón de sexo, incluido en el ámbito digital}
        \tableofcontents[hideallsubsections]
    \end{frame}


    \section{Sobre el curso}

    \subsection{Informacion general}
    \begin{frame}
        \frametitle{Información general}
        \textbf{Organización}:
        \begin{description}[Other description]
            \item[Organizador] Subdirección General de RRHH e Inspección de Servicios del MICIU
            \item[Destinatarios] Personal del MICIU y organismos dependientes
            \item[Plan de formación] 2025
            \item[Área formativa] Igualdad de Género
            \item[Duración] 9 horas
        \end{description}
        \textbf{Profesorado}:
        \begin{description}[Other description]
            \item[Mar Liñán] Responsable del Servicio de PRL del MICIU
            \item[Ignacio Cudeiro] Inspector General de Servicios del MICIU
            \item[Paz Lloria] Catedrática de Derecho penal, Universitat de València
        \end{description}
    \end{frame}

    \subsection{Antecedentes y motivación}
    \begin{frame}
        \frametitle{Antecedentes y motivación}
        \textbf{Leyes relevantes}:
        \begin{itemize}
            \item \textit{Ley Orgánica 3/2007}: Igualdad efectiva de hombres y mujeres
            \item \textit{Ley Orgánica 10/2022}: Garantía integral de la libertad sexual
            \item \textit{Ley 17/2022}: Modificación de la Ley 14/2011 de Ciencia, Tecnología e Innovación
        \end{itemize}
        \textbf{Requerimientos}:
        \begin{itemize}
            \item Informar, sensibilizar y formar al personal sobre acoso y violencia de género, incluyendo el ámbito digital y la I+D+I
            \item Planes de Igualdad de Género y Protocolos contra el acoso con seguimiento anual para agentes públicos del Sistema Español de Ciencia, Tecnología e Innovación (SECTI)
        \end{itemize}
    \end{frame}

    \begin{frame}
        \frametitle{Antecedentes y motivación}
        \textbf{Compromiso del ministerio}:
        \begin{itemize}
            \item Erradicar el acoso sexual y por razón de sexo en instituciones, universidades y centros de investigación
            \item Promover entornos laborales diversos, inclusivos, seguros e igualitarios
            \item Garantizar formación adecuada para la prevención, detección temprana y abordaje de situaciones de acoso
        \end{itemize}
        \textbf{Contexto actual}:
        \begin{itemize}
            \item Persistencia del acoso sexual y por razón de sexo en entornos laborales jerarquizados
            \item Revolución de las TIC, que ha exacerbado las desigualdades de género y la violencia contra las mujeres, incluyendo el ámbito de la I+D+I
        \end{itemize}
    \end{frame}

    \subsection{Objetivos y destinatarios}
    \begin{frame}
        \frametitle{Objetivos y destinatarios}
        \textbf{Objetivos}:
        \begin{itemize}
            \item Proporcionar formación adecuada para la prevención, detección y abordaje del acoso sexual y por razón de sexo en el ámbito laboral, incluyendo entornos digitales
            \item Destacar los desequilibrios de poder en espacios digitales y presentar la normativa existente y recomendaciones prácticas para prevenir y erradicar estas situaciones
        \end{itemize}
        \textbf{Destinatarios}:
        \begin{itemize}
            \item Todo el personal del Ministerio de Ciencia, Innovación y Universidades (MICIU) y sus organismos dependientes
            \item Especialmente dirigido a personal de RRHH, comités de igualdad, comités de selección, programas de evaluación, investigadores/as principales, y personal pre-directivo y directivo
        \end{itemize}
    \end{frame}


    \section{AS y ARS como riesgo psicosocial en el ámbito laboral}

    \subsection{Normativa}
    \begin{frame}{Normativa}
        \textbf{REVISAR}
        \begin{itemize}
            \item Normativa Internacional: Convenio 190 y Recomendación 206 de la OIT
            \item Normativa Europea: Directivas sobre igualdad de trato, conciliación de la vida familiar y profesional, y el Convenio de Estambul
            \item Normativa Española: Leyes y decretos que definen y sancionan el acoso sexual y por razón de sexo
        \end{itemize}
    \end{frame}

    \subsection{Visión del AS y ARS desde la perspectiva de la PRL}
    \begin{frame}{Visión del AS y ARS desde la perspectiva de la PRL}
        \textbf{REVISAR}
        \begin{itemize}
            \item Definición de riesgo laboral: Posibilidad de que un trabajador sufra daño derivado del trabajo
            \item Principios de la actividad preventiva: Evitar riesgos, evaluar riesgos, combatir riesgos en su origen, adaptar el trabajo a la persona, tener en cuenta la evolución de la técnica, sustituir lo peligroso, planificar la prevención, adoptar medidas de protección colectiva, y dar instrucciones a los trabajadores
            \item Factores psicosociales: Condiciones presentes en el trabajo que afectan al bienestar y salud de los trabajadores
        \end{itemize}
    \end{frame}

    \subsection{Definiciones}
    \begin{frame}{Definiciones}
        \textbf{REVISAR}
        \begin{itemize}
            \item Acoso Sexual: Comportamiento verbal o físico de naturaleza sexual que atenta contra la dignidad de una persona
            \item Acoso por Razón de Sexo: Comportamiento basado en el sexo de una persona que atenta contra su dignidad
        \end{itemize}
    \end{frame}

    \subsection{Percepción social del acoso sexual y por razón de sexo}
    \begin{frame}{Percepción social del acoso sexual y por razón de sexo}
        \textbf{REVISAR}
        \begin{itemize}
            \item Encuesta de Percepción Social de la Violencia Sexual: Proporciona una panorámica de las percepciones de la población sobre la violencia sexual
        \end{itemize}
    \end{frame}

    \subsection{Perfil de víctima y acosador/a}
    \begin{frame}{Perfil de víctima y acosador/a}
        \textbf{REVISAR}
        \begin{itemize}
            \item \textbf{Perfil de la víctima}: Mayoritariamente mujeres, especialmente aquellas en situaciones vulnerables
            \item \textbf{Perfil del acosador}: Mayoritariamente varones, pueden ser superiores jerárquicos, compañeros o clientes
        \end{itemize}
    \end{frame}

    \subsection{Consecuencias del acoso sexual y por razón de sexo}
    \begin{frame}{Consecuencias del acoso sexual y por razón de sexo}
        \begin{itemize}
            \item \textbf{Ámbito físico y fisiológico}: Enfermedades coronarias, trastornos dermatológicos, endocrinos, gastrointestinales, respiratorios, musculoesqueléticos, inmunológicos
            \item \textbf{Ámbito cognitivo}: Déficit de atención, dificultades de concentración, déficit de memoria, pensamientos recurrentes
            \item \textbf{Ámbito emocional}: Irritabilidad, ansiedad, temor, impulsividad, desconfianza, depresión
            \item \textbf{Ámbito conductual}: Sedentarismo, abuso de sustancias, conductas peligrosas, absentismo, suicidio
            \item \textbf{Ámbito social}: Aislamiento, conflictos, ausencia de comunicación
            \item \textbf{Ámbito organizacional}: Mayor absentismo, menor productividad, estancamiento laboral, despidos, responsabilidad legal
        \end{itemize}
    \end{frame}

    \subsection{Asistencia/apoyo a las víctimas de acoso}
    \begin{frame}{Asistencia/apoyo a las víctimas de acoso}
        \textbf{REVISAR}
        \begin{itemize}
            \item Apoyo emocional y acompañamiento: Información sobre derechos, asistencia psicológica y médica, asesoramiento jurídico, medidas disciplinarias y penales
        \end{itemize}
    \end{frame}

    \subsection{Detección temprana del acoso sexual y por razón de sexo}
    \begin{frame}{Detección temprana del acoso sexual y por razón de sexo}
        \begin{itemize}
            \item \textbf{Señales de alerta}: Cambios en el comportamiento de la víctima, conductas inadecuadas en el ambiente de trabajo, indicios en la comunicación digital
            \item \textbf{Estrategias de detección}: Formación y sensibilización, canales de comunicación seguros, encuestas y evaluaciones internas, supervisión y liderazgo activo, protocolos de intervención rápida
            \item \textbf{Papel de los testigos}: Observar cambios en la actitud de la víctima o del agresor, alertar de comportamientos inapropiados, apoyar a la persona afectada
        \end{itemize}
    \end{frame}

    \subsection{Prevención del acoso sexual y por razón de sexo}
    \begin{frame}{Prevención del acoso sexual y por razón de sexo}
        \textbf{REVISAR}
        \begin{itemize}
            \item \textbf{Derechos y obligaciones}: Derecho a un entorno de trabajo saludable, obligación de tratar a los demás con respeto y cooperar en la investigación de denuncias
            \item \textbf{Recomendaciones}: Potenciar un papel activo en la garantía de tolerancia cero ante el acoso, impedir el acoso, dar soporte a compañeros/as que puedan sufrir acoso
            \item \textbf{Compromiso de la empresa}: Formalizar el compromiso en documentos corporativos, generar confianza y fomentar la colaboración
            \item \textbf{Políticas de igualdad}: Estilos de gestión y organización del trabajo que dificulten el acoso, medidas específicas para garantizar la plena integración e igualdad efectiva de las mujeres
            \item \textbf{Respuesta de la organización}: Compromiso explícito contra el acoso, desaprobación de conductas ofensivas, aplicación de medidas disciplinarias severas
        \end{itemize}
    \end{frame}

    \begin{frame}{Recursos disponibles en el MICIU}
        \textbf{REVISAR}
        \begin{itemize}
            \item \textbf{Protocolos de actuación}: Adaptación del protocolo de actuación frente al acoso sexual y por razón de sexo, listado de personas habilitadas para asesoría confidencial, unidad de activación frente al acoso
            \item \textbf{Material audiovisual}: Identificar y erradicar comportamientos micromachistas, tolerancia cero frente al acoso sexual y por razón de sexo
            \item \textbf{Cursos específicos}: Plan de formación en igualdad y prevención de riesgos laborales
        \end{itemize}
    \end{frame}


    \section{Protocolo frente al acoso sexual y acoso por razón de sexo}

    \subsection{Normativa}
    \begin{frame}{Normativa}
        \begin{itemize}
            \item \textbf{Constitución Española}: Artículos que garantizan la dignidad, igualdad, integridad física y moral, y no discriminación por razón de sexo.
            \item \textbf{Directiva 2006/54/CE}: Igualdad de oportunidades y trato entre hombres y mujeres en empleo y ocupación.
            \item \textbf{Ley Orgánica 3/2007}: Igualdad efectiva de mujeres y hombres.
            \item \textbf{Ley Orgánica 1/2004}: Medidas de protección integral contra la violencia de género.
            \item \textbf{Código Penal}: Artículo 184 sobre acoso sexual.
            \item \textbf{Ley de Enjuiciamiento Criminal}: Procedimientos penales.
            \item \textbf{Estatuto de la víctima del delito}: Derechos de las víctimas.
            \item \textbf{Estatuto Básico del Empleado Público} (TREBEP): Régimen disciplinario.
            \item \textbf{Estatuto de los Trabajadores}: Derechos laborales.
            \item \textbf{Reglamento de Régimen Disciplinario}: Procedimientos disciplinarios.
            \item \textbf{Real Decreto 247/2024}: Protocolo de actuación frente al acoso sexual y por razón de sexo.
        \end{itemize}
    \end{frame}

    \subsection{Definiciones}
    \begin{frame}{Definiciones}
        \begin{itemize}
            \item \textbf{Acoso Sexual}: Solicitud de favores sexuales en el ámbito laboral, docente o de prestación de servicios, creando un entorno intimidatorio, hostil o humillante.
            \item \textbf{Acoso por Razón de Sexo}: Comportamiento basado en el sexo de una persona que atenta contra su dignidad, creando un entorno intimidatorio, degradante u ofensivo.
            \item \textbf{Otros Delitos Relacionados}: Trato degradante, coacciones, injurias y vejaciones injustas.
        \end{itemize}
    \end{frame}

    \subsection{Ámbito administrativo}
    \begin{frame}{Ámbito Administrativo}
        \framesuptitle{Medidas Organizativas}
        \begin{itemize}
            \item \textbf{Medidas Organizativas}: Formación, coordinación, comunicación, consulta y denuncia, recomendaciones de actuación, garantías y protección, prevención de riesgos laborales.
            \item \textbf{Protocolo de la AGE}: Compromiso de prevenir y no tolerar el acoso, tratamiento reservado de denuncias, identificación de responsables.
            \item \textbf{Protocolo del MCIU}: Adaptación del protocolo de la AGE, declaración institucional, definiciones, obligaciones y prohibiciones, medidas, canales de consulta y denuncia, unidad responsable, asesoría confidencial, comité de asesoramiento.
        \end{itemize}
    \end{frame}

    \begin{frame}{Ámbito Administrativo}
        \framesuptitle{Medidas Disciplinarias}
        \begin{itemize}
            \item \textbf{Principios Rectores}: Legalidad, tipicidad, irretroactividad, proporcionalidad, culpabilidad, presunción de inocencia.
            \item \textbf{Faltas}: Muy graves (discriminación y acoso), graves (abuso de autoridad, desconsideración, atentado a la dignidad).
            \item \textbf{Competencia}: Subsecretario para faltas graves y muy graves, director/presidente para faltas leves.
            \item \textbf{Sanciones}: Separación del servicio, despido disciplinario, suspensión de funciones, traslado forzoso, demérito, apercibimiento.
        \end{itemize}
    \end{frame}

    \subsection{Medidas administrativas en el ministerio de ciencia, innovación y universidades}

    \subsection{Ámbito penal}
    \begin{frame}{Ámbito Penal}
        \begin{itemize}
            \item \textbf{Jurisdicción Penal}: Denuncia de la persona agraviada, representante legal o Ministerio Fiscal.
            \item \textbf{Obligación de Denuncia}: Profesionales que tengan noticia de algún delito público.
            \item \textbf{Suspensión del Procedimiento Disciplinario}: Indicios fundados de criminalidad.
            \item \textbf{Penas}: Privativas de libertad, inhabilitación, prohibición de aproximación a la víctima, prohibición de residencia.
            \item \textbf{Unidades Especializadas}: Fiscalía de Sala de Violencia sobre la Mujer, Unidad de Coordinación de Violencia sobre la Mujer, Observatorio de Violencia Doméstica y de Género, Delegación del Gobierno contra la Violencia de Género.
        \end{itemize}
    \end{frame}


    \section{Acoso y violencia de género en el ámbito digital}

    \subsection{Normativa}
    \begin{frame}{Normativa}
        \begin{itemize}
            \item \textbf{Normativa Internacional}: Convenio 190 y Recomendación 206 de la OIT.
            \item \textbf{Normativa Europea}: Directivas sobre igualdad de trato, conciliación de la vida familiar y profesional, y el Convenio de Estambul.
            \item \textbf{Normativa Española}: Leyes y decretos que definen y sancionan el acoso sexual y por razón de sexo.
        \end{itemize}
    \end{frame}

    \subsection{El Acoso y la Violencia de Género en el Ámbito Digital}
    \begin{frame}{El Acoso y la Violencia de Género en el Ámbito Digital}
        \begin{itemize}
            \item \textbf{Sociedad Digital}: La transición de la sociedad analógica a la digital ha llevado a que los atentados contra las mujeres se reproduzcan y multipliquen en el mundo digital. La tecnología permite el control y la vigilancia de las mujeres, facilitando situaciones de acoso.
            \item \textbf{Delitos Tecnológicos}: Incluyen ciberdelitos, delitos informáticos y cibercriminalidad. Estos delitos presentan dificultades en su persecución debido al anonimato, la volatilidad de las pruebas y la descentralización del medio.
            \item \textbf{Violencia de Control}: Las TIC se utilizan como instrumentos de dominación en relaciones de pareja no sanas, incrementando el número de denuncias por violencia de género digital.
            \item \textbf{Ciberviolencia Sexual}: Incluye delitos como el maltrato habitual, el stalking, el sexting y la ciberviolencia sexual. La Memoria de la Fiscalía General del Estado (FGE) destaca la necesidad de abordar la violencia digital, especialmente entre adolescentes.
        \end{itemize}
    \end{frame}

    \subsection{Marco Normativo}
    \begin{frame}{Marco Normativo}
        \begin{itemize}
            \item \textbf{Directiva UE 2024/1385}: Aborda la ciberviolencia contra las mujeres, tipificando penalmente la violación basada en la falta de consentimiento, la mutilación genital femenina y determinadas formas de ciberviolencia.
            \item \textbf{Regulación Comparada}: Ejemplos de México y Argentina, donde se han implementado leyes específicas para abordar la violencia digital y mediática.
        \end{itemize}
    \end{frame}

    \subsection{Concepto de Violencia}
    \begin{frame}{Concepto de Violencia}
        \begin{itemize}
            \item \textbf{Violencia Machista Digital}: Comportamientos de dominación, sometimiento, control y lesión de bienes jurídicos de titularidad femenina en el espacio virtual.
            \item \textbf{Violencia Sexual Digital}: Incluye la difusión de material íntimo sin consentimiento, deepfakes sexuales y otros actos que menoscaban la integridad moral de las víctimas.
        \end{itemize}
    \end{frame}

    \subsection{Delitos de Difusión de Imágenes Íntimas}
    \begin{frame}{Delitos de Difusión de Imágenes Íntimas}
        \begin{itemize}
            \item \textbf{Art. 197.7}: Castiga la obtención y difusión de imágenes de contenido íntimo sin consentimiento. La reforma de 2022 incluye la difusión de deepfakes y la agravación de penas en casos específicos.
            \item \textbf{Art. 173.2}: Castiga la violencia habitual que incluye la difusión de imágenes íntimas.
        \end{itemize}
    \end{frame}

    \subsection{Delitos en el Ámbito Empresarial}
    \begin{frame}{Delitos en el Ámbito Empresarial}
        \begin{itemize}
            \item \textbf{Acoso Sexual}: Art. 184 del Código Penal, que establece penas para el acoso sexual en el ámbito laboral, docente o de prestación de servicios.
            \item \textbf{Acoso Predatorio}: Art. 172 ter, que incluye la utilización de la imagen de una persona para realizar anuncios o abrir perfiles falsos, ocasionando acoso, hostigamiento o humillación.
        \end{itemize}
    \end{frame}

    \subsection{Responsabilidad Penal de la Persona Jurídica}
    \begin{frame}{Responsabilidad Penal de la Persona Jurídica}
        \begin{itemize}
            \item \textbf{Art. 31 bis del Código Penal}: Establece la responsabilidad penal de las personas jurídicas por delitos cometidos en su nombre o por cuenta de las mismas, y en su beneficio directo o indirecto.
            \item \textbf{Exención de Responsabilidad}: Si la persona jurídica adopta y ejecuta modelos de organización y gestión eficaces para prevenir delitos.
        \end{itemize}
    \end{frame}

    \subsection{Conclusiones}
    \begin{frame}{Conclusiones}
        \begin{itemize}
            \item \textbf{Concepto Amplio de Beneficio}: Cualquier clase de ventaja provechosa para la persona jurídica.
            \item \textbf{Importancia de la Prevención}: Implementar la perspectiva de género en los planes de prevención y actualizar los protocolos de prevención del acoso y los planes de igualdad.
        \end{itemize}
    \end{frame}
\end{document}